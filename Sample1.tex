
\documentclass[12pt]{article}
\usepackage[utf8]{inputenc}
\usepackage[a4paper,margin=1in]{geometry}
\usepackage{xcolor}
\usepackage{listings}
\usepackage[colorlinks=true]{hyperref}
\usepackage{enumitem}
\usepackage{titlesec}

% Dracula theme colors
\definecolor{draculabg}{HTML}{282A36}
\definecolor{draculafg}{HTML}{F8F8F2}
\definecolor{draculacomment}{HTML}{6272A4}
\definecolor{draculacyan}{HTML}{8BE9FD}
\definecolor{draculagreen}{HTML}{50FA7B}
\definecolor{draculapink}{HTML}{FF79C6}
\definecolor{draculapurple}{HTML}{BD93F9}
\definecolor{draculared}{HTML}{FF5555}
\definecolor{draculayellow}{HTML}{F1FA8C}

% Page and text colors
\pagecolor{draculabg}
\color{draculafg}

% Hyperref colors
\hypersetup{
  linkcolor=draculacyan,
  urlcolor=draculagreen,
  citecolor=draculapurple
}

% Section colors
\titleformat{\section}{\Large\bfseries\color{draculapurple}}{\thesection.}{0.6em}{}
\titleformat{\subsection}{\large\bfseries\color{draculapink}}{}{0.4em}{}
\titleformat{\subsubsection}{\normalsize\bfseries\color{draculacyan}}{}{0.4em}{}

% Lists
\setlist[itemize]{itemsep=0.2em,topsep=0.2em}
\setlist[enumerate]{itemsep=0.2em,topsep=0.2em}

% Listings style
\lstset{
  backgroundcolor=\color{draculabg},
  basicstyle=\ttfamily\small\color{draculafg},
  keywordstyle=\color{draculapink},
  stringstyle=\color{draculayellow},
  commentstyle=\color{draculacomment},
  identifierstyle=\color{draculacyan},
  frame=single,
  rulecolor=\color{draculacomment},
  breaklines=true,
  showstringspaces=false,
  tabsize=2
}

\begin{document}

\begin{center}
{\LARGE \textbf{API Penetration Testing: A Comprehensive Report}}\\[0.6em]
{\large \textcolor{draculagreen}{Author: ANANYA K} \ \ \textcolor{draculacyan}{\href{https://www.linkedin.com/in/ananya-k-46184125b}{LinkedIn}}}\\[0.25em]
{\large \textcolor{draculapurple}{Platform: ROOTX}}\\[0.25em]
{\large \textcolor{draculapurple}{Date: August 20, 2025}}
\end{center}

\section{Executive Summary}
This report provides a comprehensive overview of the methodologies and practical techniques
used in API (Application Programming Interface) penetration testing. APIs serve as the
backbone of nearly all modern applications, connecting user interfaces to backend services and
databases. This ubiquitous reliance has made them a critical and often underestimated attack
surface.

The objective of this document is to detail the process of identifying, exploiting, and mitigating
vulnerabilities within APIs. It covers foundational concepts, essential security tools, and hands-
on demonstrations of the OWASP API Security Top 10 risks. The findings are based on a
practical, lab-based approach, supplemented with real-world examples from public bug bounty
programs to underscore the real-world impact of these security flaws.

By mastering the techniques outlined in this report, security professionals can significantly
enhance their ability to discover high-impact vulnerabilities and protect organizations from
data breaches and financial loss.

\section{Introduction to APIs: The New Frontier of Web Security}

\subsection*{What is an API?}
An API, or Application Programming Interface, is a set of defined rules and protocols that
allows different software applications to communicate and exchange data. In a typical web
application, the user’s browser (the client) sends requests to a server via an API, which then
retrieves or modifies data and sends a response back.

A simple analogy is a waiter in a restaurant:
\begin{itemize}
  \item \textbf{You (the user):} Place an order for a specific dish.
  \item \textbf{The Waiter (the API):} Takes your order to the kitchen.
  \item \textbf{The Chef (the backend server):} Prepares the dish.
  \item \textbf{The Waiter (the API):} Delivers the finished dish to your table.
\end{itemize}

This role makes APIs a central point of data flow and a prime target for security exploitation if
not properly secured.

\subsection*{API Attack Surface}
The attack surface of an API is vast due to its direct connection to backend data and business
logic. Common vulnerabilities found in APIs can lead to data exposure, privilege escalation,
and business disruption. Despite this, a significant number of security researchers still primarily
focus on traditional web application vulnerabilities, overlooking the massive opportunities for
bug discovery in API endpoints.

Public data from bug bounty platforms like HackerOne and Bugcrowd confirms this trend, with
hundreds of programs publicly listing their API endpoints as in-scope targets.

\subsection*{Common Types of APIs}
\begin{itemize}
  \item \textbf{REST (Representational State Transfer):} The most popular and widely adopted API
  architecture. It uses standard HTTP methods (GET, POST, PUT, DELETE) to perform
  actions on resources and relies on lightweight JSON data for communication.
  \item \textbf{SOAP (Simple Object Access Protocol):} An older, more rigid protocol that uses XML
  for data exchange. It is commonly found in legacy enterprise applications and is more
  complex to work with.
  \item \textbf{GraphQL:} A modern, flexible query language developed by Facebook. It allows
  clients to request only the specific data they need, which reduces network overhead and
  improves performance. GraphQL also uses JSON for data.
\end{itemize}

\section{Key Methodologies and Tooling}
A structured approach to API penetration testing involves a combination of manual testing and
automated tools.

\subsection*{Lab Environment}
The Vulnerable Adversely Programmed API (VAPI) was used as a safe, dedicated practice
environment. VAPI is an open-source, Docker-based application with built-in vulnerabilities
that align with the OWASP API Security Top 10 list.

\subsubsection*{Lab Setup Process:}
\begin{enumerate}
  \item \textbf{Clone the Repository:}
  \begin{lstlisting}[language=bash]
git clone https://github.com/roottusk/vapi.git
cd vapi
  \end{lstlisting}

  \item \textbf{Start the Docker Containers:}
  \begin{lstlisting}[language=bash]
docker compose up -d
  \end{lstlisting}

  \item \textbf{Access the Lab:} Once running, the API can be accessed at \texttt{http://<your\_local\_ip>/vapi}.
\end{enumerate}

\subsection*{Essential Tools}
\begin{itemize}
  \item \textbf{Postman:} A powerful, industry-standard API client. Postman was used for:
  \begin{itemize}
    \item Sending and managing API requests.
    \item Organizing API endpoints into Collections.
    \item Using Environments to manage variables like base URLs and tokens.
    \item Leveraging Postman AI to automatically generate test scripts for baseline security checks.
  \end{itemize}

  \item \textbf{Burp Suite:} A leading web proxy and security testing tool. Its primary functions in this context were:
  \begin{itemize}
    \item \textbf{Proxy:} Intercepting and analyzing all API traffic between Postman and the server.
    \item \textbf{Repeater:} Manually replaying and modifying requests to test for vulnerabilities.
    \item \textbf{Intruder:} Automating brute-force and fuzzing attacks against API endpoints.
  \end{itemize}

  \item \textbf{Neuer:} A specialized tool for API Attack Surface Mapping. Neuer analyzes
  application source code or API specification files (like Swagger/OpenAPI) to discover
  all existing endpoints, methods, parameters, and headers. This allows testers to find
  ``hidden'' or undocumented endpoints (e.g., \texttt{/v2/admin}) that might not be visible during
  normal dynamic testing.

  \item \textbf{Gron:} A command-line utility that converts JSON data into a simple, line-oriented,
  greppable format. This is crucial for parsing large, complex JSON responses and
  efficiently searching for sensitive data or interesting fields.
\end{itemize}

\section{Detailed Vulnerability Analysis and Practicals}

\subsection*{API1: Broken Object Level Authorization (BOLA)}
\textbf{Description:} This vulnerability occurs when an API endpoint accepts a user-supplied object
ID (e.g., \texttt{/users/123}) but fails to verify that the requesting user is authorized to access that
specific object. An attacker can simply change the ID in the request to view or modify other
users' data. BOLA is the API equivalent of a classic IDOR (Insecure Direct Object Reference)
bug.

\textbf{Practical Demonstration:}
\begin{enumerate}
  \item \textbf{Register a User:} A new user is created via the \texttt{POST /api/v1/user} endpoint. The server
  assigns an ID, for example, \texttt{id: 7}.
  \item \textbf{Access Own Data:} A \texttt{GET /api/v1/user/7} request successfully returns the user's data.
  \item \textbf{Exploitation:} The attacker changes the ID in the GET request to \texttt{1}, which is the ID of
  another user (in this case, an administrator account).
  \item \textbf{HTTP}
  \begin{lstlisting}
GET /api/v1/user/1
  \end{lstlisting}
  \item \textbf{Result:} The API returns the sensitive data of user \texttt{id: 1}, including a flag that confirms
  the vulnerability.
\end{enumerate}

\textbf{Remediation:} Always implement robust authorization checks on the server side. Before an
API processes a request for a resource, it must verify that the authenticated user owns or has
explicit permission to access that resource.

\subsection*{API6: Mass Assignment}
\textbf{Description:} Mass assignment allows an attacker to add extra, un-intended parameters to an
API request. If the backend automatically maps all received parameters to a database object
without a strict whitelist, the attacker can modify sensitive attributes like \texttt{isAdmin}, \texttt{role}, or
\texttt{credits}.

\textbf{Practical Demonstration:}
\begin{enumerate}
  \item \textbf{Normal User Creation:} A user is created with a username, name, and password. The
  user's account is created with \texttt{credits: 0}.
  \item \textbf{Exploitation:} The attacker sends a new \texttt{POST} request to create a user but adds an extra
  parameter, \texttt{"credit": 100}, to the request body.
\end{enumerate}

\textbf{JSON}
\begin{lstlisting}[language=json]
{
  "name": "New Hacker",
  "username": "new_hacker",
  "password": "pass123",
  "credit": 100
}
\end{lstlisting}

\textbf{Result:} The API processes the request and assigns \texttt{100} credits to the new account,
demonstrating that the backend did not filter out the unauthorized parameter.

\textbf{Remediation:} Developers should use a strict whitelist of allowed parameters. Any parameters
not on the list should be ignored or rejected by the API. Using Data Transfer Objects (DTOs)
with explicit field definitions can effectively prevent this.

\subsection*{API4: Lack of Resources \& Rate Limiting}
\textbf{Description:} This vulnerability occurs when an API endpoint does not have any controls to
limit the number of requests from a user or IP address over a period of time. This absence of
rate limiting makes the API vulnerable to brute-force attacks and denial of service.

\textbf{Practical Demonstration:}
\begin{enumerate}
  \item \textbf{Identify a Target:} A mobile login API for a \texttt{v2} endpoint is used. It requires a 4-digit
  PIN for authentication.
  \item \textbf{Brute-Force Attack:} The request is sent to Burp Suite's Intruder. A payload set is
  configured to test all possible numbers from \texttt{0000} to \texttt{9999}.
  \item \textbf{Result:} The server does not block or rate-limit the requests. The brute-force attack
  successfully finds the correct PIN, leading to an account takeover.
\end{enumerate}

\textbf{Real-World Examples:}
\begin{itemize}
  \item A well-known bug bounty report detailed a security researcher who earned a \${}15{,}000
  bounty from GitHub by brute-forcing a 6-digit OTP due to a lack of rate limiting on
  the password reset API.
  \item Another researcher received \${}30{,}000 for a similar brute-force vulnerability on an
  Instagram API.
\end{itemize}

\textbf{Remediation:} Implement server-side rate-limiting for all authentication-related endpoints.
This includes login, password reset, and OTP verification APIs.

\subsection*{API7: Security Misconfiguration (CORS)}
\textbf{Description:} Cross-Origin Resource Sharing (CORS) is a mechanism that allows or denies
web pages from making cross-origin requests. A misconfigured CORS policy, such as allowing
access from \texttt{*} (any origin) or reflecting an attacker's domain in the \texttt{Access-Control-Allow-Origin}
header, can allow a malicious website to steal sensitive data from a vulnerable API.

\textbf{Practical Demonstration:}
\begin{enumerate}
  \item \textbf{Find a Vulnerable Endpoint:} A legitimate user logs in and fetches a ``personal key''
  via the API.
  \item \textbf{Check for Misconfiguration:} The attacker sends a request with a custom header,
  \texttt{Origin: evil.com}.
  \item \textbf{Result:} The server responds with \texttt{Access-Control-Allow-Origin: *} and \texttt{Access-Control-Allow-Credentials: true}. The \texttt{*} indicates that the API is willing to share data with any
  domain, and \texttt{true} allows the attacker to steal the user's sensitive data (like cookies or
  API keys) from the response.
\end{enumerate}

\textbf{Remediation:} Configure the \texttt{Access-Control-Allow-Origin} header to only list explicitly
trusted domains. Avoid using a wildcard (\texttt{*}) and set \texttt{Access-Control-Allow-Credentials} to \texttt{false}
unless absolutely necessary.

\section{Advanced Techniques and Tools}

\subsection*{API Fuzzing}
Fuzzing is a powerful technique for discovering new vulnerabilities. A simple bash script can
be used to send a list of pre-defined payloads (e.g., SQL injection, XSS) to multiple API
endpoints.

\textbf{Fuzzer Script:} The fuzzer script automates the process of testing endpoints against payloads.
It can detect vulnerabilities by looking for specific keywords in the server's response (e.g.,
``SQL,'' ``error,'' or the reflected payload itself). Critically, the script is configured to send all
requests through a proxy like Burp Suite, allowing the tester to review and manually modify
any suspicious-looking requests.

\textbf{Case Study: WAF Bypass:} Fuzzing can reveal how an application handles unusual input. For
example, a fuzzing-based approach can uncover payloads that bypass a Web Application
Firewall (WAF) like Cloudflare, allowing for successful exploitation of a cross-site scripting
(XSS) vulnerability.

\subsection*{Attack Surface Mapping with Neuer}
When source code or API documentation is available (a white-box scenario), tools like Neuer
can be invaluable. Neuer analyzes the source code of an application to automatically identify
all API endpoints, methods, and parameters.

\textbf{Benefits:}
\begin{itemize}
  \item \textbf{Comprehensive Coverage:} Finds ``hidden'' or undocumented endpoints that a tester
  might not discover through normal browsing.
  \item \textbf{Efficient Fuzzing:} Provides a complete list of parameters, ensuring that the fuzzer tests
  every possible input field.
  \item \textbf{Time-Saving:} Drastically reduces the time spent on manual reconnaissance, allowing
  the tester to focus on exploitation.
\end{itemize}

\section{Conclusion and Future Directions}
This report has detailed a comprehensive approach to API security testing, covering a range of
vulnerabilities from object-level authorization bypasses to advanced fuzzing techniques. By
understanding API fundamentals, utilizing the right tools, and applying a methodical approach,
security professionals can effectively protect a company's most critical assets.

The skills covered—including BOLA, Mass Assignment, Rate Limiting, CORS, and the use of
tools like Postman, Burp Suite, and Neuer—form a robust foundation. Continued learning and
practice are essential to keep pace with evolving API technologies and emerging threats.

\end{document}
