\documentclass[11pt]{article}
\usepackage[utf8]{inputenc}
\usepackage[margin=1in]{geometry}
\usepackage{xcolor}
\usepackage{hyperref}
\usepackage{titlesec}
\usepackage{enumitem}
\usepackage{booktabs}
\usepackage{graphicx}

% Dark theme colors
\definecolor{darkbg}{RGB}{40, 40, 40}
\definecolor{lighttext}{RGB}{240, 240, 240}
\definecolor{accent}{RGB}{0, 150, 200}

% Page setup
\pagecolor{darkbg}
\color{lighttext}

% Section formatting
\titleformat{\section}{\color{accent}\Large\bfseries}{\thesection}{1em}{}
\titleformat{\subsection}{\color{accent}\large\bfseries}{\thesubsection}{1em}{}

% List formatting
\setlist[itemize]{itemsep=0pt, topsep=5pt}
\setlist[enumerate]{itemsep=0pt, topsep=5pt}

% Hyperlink setup
\hypersetup{
    colorlinks=true,
    linkcolor=accent,
    urlcolor=accent,
    citecolor=accent
}

\begin{document}

\section*{HOWTO MONETIZE YOUR CYBERSECURITY KNOWLEDGE}
\subsection*{Learn how to make some \$\$ from your cybersecurity knowledge}
\subsection*{TAIMUR IJLAL}

\section*{Table of Contents}
\begin{itemize}
\item Introduction \dotfill 2
\item How to use this eBook \dotfill 7
\item Writing on Medium \dotfill 10
\item Start a YouTube Channel \dotfill 18
\item Create a Udemy course \dotfill 23
\item Publish an eBook on Gunroad \dotfill 41
\item Self-publish a book on KDP \dotfill 47
\item Freelancing on Fiverr \dotfill 53
\item 1-1 Bookings \dotfill 58
\item Self-Hosted Courses \dotfill 63
\item Time to Earn \dotfill 67
\end{itemize}

\section*{Introduction}
Money! Everyone wants more of it due to the happiness and stability it brings to your life (plus all the cool stuff you can buy on Amazon 😊)

Cybersecurity is one of the most in-demand professionals today but like anyone else; cyber-security pros can do with some more money in the bank.

\section*{Why I wrote this book}
Every year there are thousands of books and articles written on how to make more income given the popularity of this topic. BUT .. none of them are from the perspective of a cybersecurity professional. So, I thought it would be useful to write something more focused on the cybersecurity industry and how to monetize this knowledge.

Despite the glitz and glamor of cybersecurity, a lot of people much more qualified than me make peanuts and work 60 to 80 hours a week. They are entirely dependent on their monthly salary for any type of income.

\subsection*{The purpose of this book is simple .. to help Cybersecurity pros make more money outside of their 9 to 5 jobs by starting simple side hustles that can grow over time.}

Nearly every recommendation I am giving you in this book, I have tried myself, and believe me when I say it can be life-changing if you put in the work.

You will see money coming in that you never expected to see.

Having additional income streams enables you to escape the paycheck-to-paycheck lifestyle, achieve financial independence and diversify your sources of income which can be a massive relief if you find yourself out of a job for any reason.

\section*{About me}
My name is Taimur Ijlal, and my profile on LinkedIn says something like:

\textit{A multi-award-winning, information security leader with over 21+ years of international experience in cyber-security and IT risk management in the fin-tech industry. Strong knowledge of ISO 27001, PCI DSS, GDPR, Cloud Security, DevSecOps, etc. Also a published writer, speaker, and course creator on Cloud Security and Artificial Intelligence...}

The truth is that I lucked out a lot in my career due to many people guiding and helping me. I have worked in cybersecurity for a few decades but these last couple of years I have focused more on giving back to the cybersecurity community.

To do that, I have written books, created courses, started a YouTube channel, blogs, etc.

If you want to get in touch with me, then you can connect with me below:

\begin{itemize}
\item LinkedIn
\item My Free Newsletter
\item YouTube Channel
\item Udemy
\item 1-1 Consultations (not free!)
\end{itemize}

If you want to drop me a message then LinkedIn is usually the best way to do it.

\section*{How to use this eBook}
This is not a book on financial advice as I am not qualified to give that.

Instead, this book is full of simple straightforward advice on how you can monetize your cybersecurity knowledge via simple side hustles that ANYONE can start.

All of the tips I have recommended are ideas that I have experimented with myself to varying degrees of success.

I will share with you the results I made, and you could easily make more (or less) than me depending on how much effort you put in.

If you feel like you cannot do these things then remember that all of the things I am mentioning in this book, I did without:

\begin{itemize}
\item Doing any paid advertising
\item Having a million followers on TikTok or Instagram
\item Doing any webinars
\item Having an email list of thousands of subscribers.
\end{itemize}

These are not "get rich quick" ideas as most of those on the internet are fake / scams.

You can get rich if you put in the hard work over a period of time and let these income streams gather momentum.

Put in the hard work now and this effort will compound over time. You definitely will see the benefits after a while and feel grateful that you started.

The amount of money you can earn via these methods differs from person to person and depends on the amount of effort you are willing to put in. There is honestly no ballpark figure I can give for each stream as it directly depends on your unique skills and your circumstances.

If you want my advice, go through these methods one by one and do not jump ahead to whatever income stream looks interesting to you. I have made these from easy to hard and each stream builds on the previous one.

\subsection*{Remember the ice cube analogy}
It is a sad but true fact that most people give up on their side hustles after trying them for a few months and not seeing any result.

\textbf{I understand as It can get very frustrating not seeing any engagement after you put hours into making a YouTube video or a book or a digital course.}

When I started, I often found myself questioning if I had not wasted all those hours and effort when I did not see any results coming in.

This is why the ice cube analogy in the book "Atomic Habits" really resonated with me.

The book simply states that if you are in a cold room with an ice cube and the room slowly starts getting warmer, then you will not see any change from 25 degrees to 31 degrees, but suddenly, at 32 degrees, the ice cube will start to melt.

\begin{quote}
"Complaining about not achieving success despite working hard is a bit like complaining about an ice cube not melting when you heated it from 25 to 31 degrees. Your work was not wasted; it is just being stored ... All the action happens at 32 degrees" — James Clear
\end{quote}

This simple analogy gave me a huge boost when I realized that all my previous efforts had not been wasted.

They were compounding over time, and even the worst side hustle had helped me improve over time.

Now I approach every new side hustle with enthusiasm, knowing that even if not a single person buys or watches it; the effort of creating it has not been wasted and will pay off over time.

Keep this in mind and you will not get demotivated.

\subsection*{LET'S START!}

\section*{Writing on Medium}
\textbf{DIFFICULTY - LOW}

The first way to monetize your Cybersecurity knowledge is to start writing on Medium.

Follow these steps:

\begin{enumerate}
\item Go to www.medium.com
\item Create an account
\item Write an article about a subject in tech or cybersecurity that you are good at
\end{enumerate}

Medium, if you are not aware, is a blogging website and one of the best ways for cybersecurity professionals to a) create a personal brand and b) make some money.

Provided you write consistently good articles which add value; Medium can be an awesome place to build and earn some of that sweet, sweet money AND send traffic to your other side hustles (more on that later).

Writing on Medium is also the most accessible side hustle to start.

Unlike YouTube or Udemy, which require speaking, video creation, editing, etc., all writing requires you to do is sit in front of a computer and start writing!

NOTE: An added benefit is that It opens you up to job opportunities in a way your standard profile will never do. Instead of waiting for someone to contact you on LinkedIn, your writing on a platform like Medium goes all over the Internet, and one of them could be your future manager.

\subsection*{How to earn money on Medium?}
There are two ways to earn money on Medium:

\begin{enumerate}
\item Directly via the Medium Partnership Program (MPP).
\item Indirectly via using Medium as a traffic course for your other side-hustles
\end{enumerate}

Let's see each of them.

\subsubsection*{1. Medium Partnership Program (MPP)}
Once you start writing on Medium, you can join the Medium Partnership program after you meet the below criteria which are basically to be a paying member (around USD 5 a month although this used to be free), write often AND be in an eligible country. \url{https://medium.com/membership}

\begin{itemize}
\item \textbf{You're a member.} The best way to succeed as a Partner Program author is to also participate as a reader of stories on Medium.
\item \textbf{You've published a story in the last 6 months.} Staying active, publishing regularly, and being engaged with the community are important ways to help our platform flourish.
\item \textbf{You're located in an eligible country.} We recently expanded the number of supported countries, so now more people than ever are able to join the Partner Program.
\end{itemize}

Once you have joined the program, you can put your articles behind a paywall and earn money. You are paid based on how much your articles are read so the more people read and engage with your articles, the more money you can earn.

People have earnt thousands of dollars from just one article (although that is quite rare!).

\subsubsection*{What if the MPP is not available in your country?}
Unfortunately, a lot of people ping me and mention this program is not available in their country which means they cannot join the Medium Partnership Program.

Do not fret as we can still use Medium for the next point which is as a traffic source.

\subsubsection*{2. Using Medium as a Traffic course for your other side hustles}
Medium is one of the most visited websites in the world, especially by tech and cybersecurity professionals. You can use Medium as a traffic source and point readers to your other side hustle like your courses and eBooks which we will create in later chapters.

So even if you do not get enrolled in the partnership program, Medium can still help you make some serious money!

\subsection*{Common excuses people make}
Despite all these advantages, I have seen many people make excuses for not starting on Medium. Some of the common ones are below.

\subsubsection*{Excuse 1: "I am not a writer!"}
Yes, and neither was I. The only thing that makes you a writer is writing so start!

\subsubsection*{Excuse 2: "It's hard" (Spoiler alert: yes, it is)}
Yes, technical writing is hard. In fact, any writing is hard at the beginning.

The more you work your writing muscle, the stronger it will get and the easier it will become.

\subsubsection*{Excuse 3: "My English is not good enough."}
I could have bought this excuse a few years back, but nowadays, with tools like Grammarly present, there really is no excuse. No one is expecting you to be the William Shakespeare of technical writing, but as long as you can put together a few sentences properly .. tools like ChatGPT and Grammarly can carry you the rest of the way.

\subsubsection*{Excuse 4: "People will make fun of me!"}
Yes, there will always be people who are insecure about anyone else trying new things and will try to drag you down .. so do not let them.

In my experience, people will continue making fun of you .. until the point they see you making money, and then they will be falling over themselves asking you for advice.

\subsection*{My earnings from Medium as of December 2023}
For full transparency, these are my Medium earnings as of December 2023. The numbers have fluctuated quite a bit due to Medium making changes to their algorithm which have impacted tech articles. BUT this is still a nice amount of money to get every month.

\begin{center}
\begin{tabular}{lr}
\toprule
\textbf{Completed periods} & \textbf{Amount} \\
\midrule
Nov 1 – Nov 30 & \$221.60 \\
Oct 1 – Oct 31 & \$362.16 \\
Sep 1 – Sep 30 & \$269.14 \\
Aug 1 – Aug 31 & \$256.55 \\
\bottomrule
\end{tabular}
\end{center}

\subsection*{A gift for you}
If you still are not familiar with Medium and not sure how to start, then I have a gift for you. Here is a free link to my course "Medium for beginners - How to succeed on Medium Platform."

\subsection*{Medium for beginners - How to succeed on Medium platform}
\begin{center}
89:00 \$0+ | Talmur Bial | O'satings
\end{center}

\subsection*{ChatGPT?}
A lot of people are going crazy about ChatGPT and I recommend using it for getting article ideas and outlines. DO NOT copy paste complete articles from ChatGPT into Medium as they usually get detected and downgraded. Plus, your own brand and reputation will suffer in the long run.

\subsection*{Action items}
These are the action items I need you to take:

\begin{itemize}
\item Create a Medium account and set up your profile
\item Decide on at least 5 to 7 topics to write about
\item Take my free course if you are not sure how to use Medium
\item Start posting at least once a week (or more if you can)
\end{itemize}

\section*{Start a YouTube Channel}
\subsection*{DIFFICULTY - HIGH + LOW (explained below)}
The next step is to start a YouTube channel around Cybersecurity which is another excellent way to monetize your knowledge.

YouTube is the second biggest search engine in the world, and billions of people visit it every day to find answers.

You \textbf{NEED} a presence on this platform if you are serious about making money down the road.

\subsection*{How to earn money on YouTube?}
There are two ways to earn money on Medium:

\begin{enumerate}
\item Direct via YouTube AdSense (HARD)
\item Indirect via using YouTube as a traffic source (EASY)
\end{enumerate}

Let's look at each of them.

\subsubsection*{1 - Direct way via YouTube AdSense - THE HARD WAY}
Starting a YouTube channel is pretty straightforward and free however getting it monetized requires a lot of effort and patience. A YouTube channel needs over 1000 subscribers and 4000 hours of watch time before it can be monetized, so consistency and patience is the key.

You need to regularly put out engaging content that people will want to watch over a period of time before you will start to see the benefits.

YouTube is an EXTREMELY competitive space and in order to stand out amongst the billions and billions of videos already present you need to have good presentation and editing skills, good audio and video setup, and an ability to speak on a variety of topics.

It can take months or even years before you see money from YouTube however do not let that discourage you.

\subsubsection*{2 - Indirect way via using YouTube as a traffic source - THE EASY WAY}
Even a non-monetized YouTube Channel can become an amazing source of income for you.

Just like Medium, YouTube is absolutely amazing in sending traffic to your other income sources like your blog and online courses. We will discuss this in later sections.

\subsection*{What type of content to make?}
Choose whatever cybersecurity or tech topic you are comfortable with and start making videos.

You do not need any hi-tech equipment, and just your phone will suffice for recording audio and video.

You can also use free software like Canva for creating slides or Zoom for recording.

\textbf{One easy tip for creating content is to convert the articles you write on Medium into videos. The content and script are already there, and you just have to change the format.}

\subsection*{My earnings from YouTube as of December 2023}
From YouTube AdSense I have earned around (drum roll ...)

\subsection*{20 USD per month}
Yes sorry to disappoint but as of Dec 2023 my tiny channel just has around 4K subscribers and generates around 20 USD per month. I guess I am not going to be retiring to the Bahamas just yet!

\begin{itemize}
\item \textbf{Views} 8.5K 216 more than usual
\item \textbf{Watch time (hours)} 401.1 11.1 more than usual
\item \textbf{Subscribers} +200 About the same as usual
\item \textbf{Estimated revenue} \$23.46 About the same as previous 28 days
\item \textbf{Mon, 4 Dec 2023} \$0.59
\end{itemize}

Yet despite this, YouTube has been an amazing way of sending traffic to my other products and increasing my earnings. You will not believe the number of people who have purchased my courses or set up paid 1-1 calls with me because of my videos.

\subsection*{ChatGPT ?}
ChatGPT is great for:

\begin{itemize}
\item Getting video ideas. Prompt it for good ideas about videos that people might watch regarding cybersecurity and refine them further
\item Generating YouTube Scripts based on the idea you choose
\item Writing video descriptions
\item Giving you ideas for YouTube thumbnails
\end{itemize}

\subsection*{Action items}
These are the action items I need you to take:

\begin{itemize}
\item Create a YouTube account
\item Create 2 to 3 videos on YouTube. You can see the same topic from your Medium article
\item Start posting at least once a week (or more if you can). Try to reach 35 videos as soon as you can so the Youtube algorithm notices you.
\end{itemize}

\section*{Create a Udemy course}
\subsection*{DIFFICULTY - EASY}
Online courses are all the rage nowadays and Udemy is one of the most popular platforms for cybersecurity professionals to upskill themselves. Millions of learners are present and you can tap into this user base by creating and selling a course.

A broad selection of courses

Choose from 213,000 online video courses with new additions published every month.

Python Excel Web Development JavaScript Data Science AWS Certification Drawing

Expand your career opportunities with Python

Take one of Udemy's range of Python courses and learn how to code using this incredibly useful language. Its simple syntax and readability makes Python perfect for Flask, Django, data science, and machine learning. You'll learn how to build everything from games to sites to apps. Choose from a range of courses that will appeal to...

Explore Python

Learn Python: The Complete Python Programming Course
Accessible: The Course
4.5 + 9 + 8 + 9 (0.375)
£59.00

Learning Python for Data Analysis and Visualization Ver 1
Java Profile
4.5 + 9 + 8 + 9 (0.375)
£59.00

Python for Beginners: Learn Programming from scratch
EAN-Data: College Faculty Solutions
4.5 + 9 + 8 + 9 (0.395)
£59.00

Learn Python: Python for Beginners
Albert Hastie
4.5 + 9 + 8 + 9 (0.375)
£59.00

Python From Scratch \& Selection: WebDriver Q\&A...
Accessible:
4.5 + 9 + 8 + 9 (0.375)
£59.00

The best thing is that Udemy does all the promotional work for you, so you just need to create a good course and let the website handle all the rest.

\subsection*{UDEMY = AMAZON for online courses}
Very little effort is required for your first course as you don't need to create a four-hour long course on your first try.

The minimum length of a Udemy course is around 30 minutes ! That's around the length of a few TikTok videos honestly.

All you need is PowerPoint and a free Zoom account to start.

\subsection*{How to create a Udemy course ?}
You need to find a cybersecurity topic that has high demand and low competition i.e. something which people are looking for, but there are not enough courses present.

Follow these steps:

\subsubsection*{1 — Create an instructor account}
Go to \url{https://www.udemy.com/teaching/} and create your instructor account.

It is completely free. Just answer a few basic questions and have your account ready.

\subsubsection*{2 — Know the tools Udemy gives you}
Udemy gives you some great tools to get started. Go to the insights section.

We are interested in the Marketplace insights that will let us know which cybersecurity topics are in demand right now.

\subsubsection*{3 — Find an awesome cybersecurity topic}
This is the most important step to focus on and where most people make mistakes.

You have to choose a topic that is \textbf{HIGH} in demand but \textbf{LOW} in competition (this will be familiar to anyone that knows SEO).

Let's take a look at a few topics.

If you choose a generic topic like \textbf{Cybersecurity} then there is not much chance for us to get our course to succeed.

Opportunity overview for English courses on Cyber Security

Aim for high ratings to succeed in this topic

Create Your Course

Student demand: high

Number of courses: high

Median monthly revenue: \$26 per month

Top monthly revenue: no data available

Ok .. how about "Cloud Security" .. same result.

Opportunity overview for English courses on Cloud Security

Bring your "A" game to succeed in this topic

Create Your Course

Student demand: high

Number of courses: high

Median monthly revenue: \$48 per month

Top monthly revenue: \$443 per month

Same with a security certification topic .. CISSP gives me the below.

Opportunity overview for English courses on \textbf{CISSP - Certified Information Systems Security Professional}

Bring your "A" game to succeed in this topic

Create Your Course

Student demand: high

Number of courses: high

Median monthly revenue: \$13 per month

Top monthly revenue: — — no data available

Read to give up ??

Let's narrow it down and choose a more specialized topic.

Instead of Cloud security .. Let me pick "AWS security".

Opportunity overview for English courses on AWS Security Services

This topic is a great opportunity!

Create Your Course

Student demand: high

Number of courses: low

Median monthly revenue: \$32 per month

Top monthly revenue: \$239 per month

BOOM! we have our first topic. Students want this course and the number of courses is low also makes it a great topic to teach on.

lets's look at a bit more .. I love the topic of DevSecOps and the topic is quite hot right now also.

Opportunity overview for English courses on DevSecOps

This topic is a great opportunity!

Create Your Course

Student demand: high

Number of courses: low

Median monthly revenue: \$368 per month

Top monthly revenue: \$835 per month

YES! one more topic we can target.

What about NIST cybersecurity which is something a lot of companies are interested in??

Opportunity overview for English courses on NIST Cybersecurity Framework (CSF)

This topic is a great opportunity!

Create Your Course

Student demand: high

Number of courses: low

Median monthly revenue: \$86 per month

Top monthly revenue: \$1,047 per month

So just from this, you have found three cybersecurity topics that you can start teaching TODAY and get success.

Try to find more topics that might interest you, and let me know in the comments section what you find!

if they are good I might steal them :)

\subsubsection*{4 — Mindmap your course}
Another important step that sometimes gets skipped out.

Before you start creating content.. mindmap your course with regards to the topics and sections.

How much depth are you going to go? How many lessons?

Without a structure in place, you will find it difficult to focus and either create too much or too little content.

Remember that as per Udemy:

"To be approved for the Udemy marketplace, courses must have a minimum of five lectures and at least thirty minutes of video content"

That means you just need to create 5 lessons each of 6 minutes which is quite easy!

If I was teaching AWS security services I could just create a simple mindmap like the one below to focus my efforts.

\begin{itemize}
\item a. What is AWS
\item b. How is AWS security different
\item c. AWS security concepts
\item AWS security services
\item d. Quick demo of AWS security tools
\item e. Assignment
\item f. How to do an AWS security review
\end{itemize}

\subsubsection*{5 — Start creating your content}
This is the easiest part, honestly as you just have to start creating your content based on the mindmap you did earlier.

No fancy stuff needed .. PowerPoint or Canva will be good enough for the content and your laptop for audio recording.

In Canva you can create a presentation and use the "Present and Record" feature which will allow you to show your face as you teach.

Students to prefer to see the face of their teacher when presenting so I would recommend that if you are comfortable.

You can record and then download the recording from Canva.

Once you have recorded your first lesson; be sure to see one of the other tools which Udemy gives you which is the "Test Video".

You can upload your first lesson and the Udemy team will give you free feedback on how good the audio and video quality is which is pretty awesome!

\subsubsection*{Tools}
\begin{itemize}
\item \textbf{Test Video} Get free feedback from Udemy video experts on your audio, video, and delivery.
\item \textbf{Marketplace Insights} Get Udemy-wide market data to create successful courses.
\item \textbf{Bulk coupon creation} Create multiple coupons at once via CSV upload.
\end{itemize}

\subsubsection*{7 — Edit}
Make sure to edit your raw videos as you download them. There will be lots of space and gaps and "ums" and "ahs" in your videos which you do not want in the final course.

There are free video editors available for both Windows and Macbook, which you use.

Basic video editing barely requires any skill .. if you can edit a TikTok video .. you can edit a 6-minute lesson!

\subsubsection*{8 — Upload and Launch}
Time to upload your lessons onto Udemy.

Once you click on Create a Course .. you can start translating your mind map into lessons and sections and upload your videos.

Udemy makes it incredibly intuitive and easy to do.

Plan your course
\begin{itemize}
\item Intended learners
\item Course structure
\item Setup \& test video
\end{itemize}

Create your content
\begin{itemize}
\item Film \& edit
\item Curriculum
\item Captions (optional)
\end{itemize}

Publish your course
\begin{itemize}
\item Course landing page
\item Pricing
\item Promotions
\item Course messages
\end{itemize}

Submit for Review

Curriculum

Start putting together your course by creating sections, lectures and practice (quizzes, coding exercises and assignments).
If you're intending to offer your course for free, the total length of video content must be less than 2 hours.

Section \#: D Introduction

Lecture \#: D Introduction

+ Content

Unpublished Section: D AWS security

One important thing to remember is to create a good thumbnail which immediately grabs the attention of a potential customer.

Course image

Dimensional-based videos

Upload your course image here. It must meet our course image quality standards to be accepted. Important guidelines:
750x422 pixels; .jpg, .jpeg, .gif, or .png, no text on the image.

No file selected

Upload File

Udemy needs to be 750x422 pixels with no text on the image. You can put your face or a compelling image with icons.

This is easily done via Canva, along with the dimensions.

Choose "Create a design" in Canva along with Custom Size and you can create a nice thumbnail for your course.

If you are feeling very ambitious you can create a promotional video of around a minute or two telling customers how awesome your course is.

It is not mandatory but increases enrollments by over 5 times as per Udemy.

Once it is done, you can submit your course to Udemy for review.

They usually approve within a day for two, and congrats as you have your first digital course live on Udemy!

\subsubsection*{9 — Time to get those reviews}
Your work is not done yet.

In order for your course to stand out you need to get those reviews and ratings on your course.

This is quite important as the more reviews you get the higher you will rank in Udemy search and the algorithm will start noticing you.

Get in those initial reviews and the Udemy algorithm will do the rest.

There are easy ways to do this even if you do not have a email list or a large social media following.

First step is to generate some free coupons on Udemy for your course. You can access this from the Promotions section on your Course page.

Create a new coupon
You can create 1 new coupon until the end of October

Pick a coupon type

Current best price
\$9.99
Unlimited redemptions
Expires in 5 days

Custom price
Between \$12.99 and \$19.99
Unlimited redemptions
Expires in 31 days

Free: Open
1000 redemptions
Expires in 5 days

Free: Targeted
100 redemptions
Expires in 31 days

n-platform/

Active/Scheduled coupons

Spread this to your colleagues and network and ask them for ratings and reviews.

There are Facebook pages also dedicated to free Udemy courses but I usually have seen that people snatch up those coupons and never bother to take the course OR give a review.

Reaching out directly to your contacts is much better and gives you more organic reviews and ratings.

\subsection*{Common excuses people make}
Similar to Medium, a lot of times people make excuses for not starting on Udemy. These are the common ones.

\subsubsection*{Excuse 1 — "I am not an instructor/teacher "}
Yeah, and the only thing that will make you an instructor or teacher is to start teaching.

There is absolutely no reason you cannot take some of your cybersecurity knowledge and start spreading it, so do not let the negativity bring you down.

\subsubsection*{Excuse 2 — "There is too much competition"}
I have been hearing this for the past 3 years or so that Udemy is saturated and yet still people are making thousands of dollars every month. It is true that there are literally millions of courses on cybersecurity, and your course might get lost in the sea but you can avoid that by being smart on what topic you choose.

\subsubsection*{Myth 3 — "I don't have equipment"}
I have launched over 5 courses using nothing but my laptop and the free version of Canva, and I have reached over 20K students on Udemy. I have one of the highest rate courses, so... no excuses!

Do not get me wrong as I am not saying making money via Udemy is EASY but it is very much POSSIBLE provided you put in the hard work.

So let us start step by step from the beginning and get you on the path to earning!

\subsection*{My earnings from Udemy}
Lets see my Udemy earnings for the past few months:

\begin{itemize}
\item Mar 2023 \$374.66
\item Feb 2023 \$325.22
\item Jan 2023 \$299.81
\item Dec 2022 \$208.91
\item Nov 2022 \$141.59
\end{itemize}

As you can see, Udemy has consistently sent me over revenue which has slowly been increasing every month on a consistent basis. This is pretty awesome considering I have put in the effort once and the money has been coming in on auto-pilot since then.

\subsection*{ChatGPT ?}
ChatGPT is great for:

\begin{itemize}
\item Getting more ideas for Udemy courses
\item Generating outlines for Udemy courses which you can use to mind map further
\item Writing your course descriptions. Mention that this is a Udemy course so it will refine it further
\end{itemize}

\subsection*{Udemy Business}
One added bonus that can happen if you consistently make courses is that they may get selected for Udemy Business. This is a list of curated courses that Udemy handpicks for businesses across the world. Your course may end up being shown to thousands of employees!

NOTE: The payment structure for Udemy Business is a bit different. As they state "On Udemy Business, instructors are compensated based on learner engagement. This differs from the Udemy marketplace model, where instructors are compensated based on course purchases". Simply put the more minutes Udemy Business users watch you course ... the more you earn!

Getting into Udemy Business is not a secret. Your course needs to be of a high quality following the tips I mentioned PLUS also be a topic that is in demand in Udemy Business.

You do not have to guess this as Udemy provides a separate tool for searching Udemy Business topics in demand which you can access from the insights section:

\subsection*{Insights}
\textbf{Udemy Business content opportunities}

View content opportunities based on demand from professional learners and their employers.

Once inside, you can also filter based on "Cybersecurity" to find out which topics are trending. This can increase your chances of making a course that eventually gets selected by Udemy Business!

\subsection*{Action items}
These are the action items I need you to take:

\begin{itemize}
\item Create a Udemy instructor account
\item Research cybersecurity course topics and find one with high demand and low competition
\item Create a course outline of at least 5 lessons of 6 minutes each
\item Record your course and upload to Udemy
\item Get reviews from your contacts
\item Put the course announcement on your Medium and YouTube channels
\item Ping me also and I will be happy to take your course and leave some honest feedback
\end{itemize}

\section*{Publish an eBook on Gunroad}
\subsection*{DIFFICULTY - EASY}
If you have been following my advice, you should have created a Udemy course, written a few articles on Medium and made a few YouTube videos. If those were around a common cybersecurity topic, then why not add one more income stream and create an eBook you can sell on Gunroad?

\subsection*{What is Gunroad?}
\textbf{Gunroad} is a \textit{"self-publishing digital marketplace platform to sell digital services such as books, memberships, courses, and other digital services"}. It is pretty easy to use and you can create Ebooks, Digital Courses, Membership programs, etc. Below are the types of things you can sell.

\begin{itemize}
\item \textbf{Digital product} Any set of files to download or stream.
\item \textbf{Newsletter} Deliver recurring content through email.
\item \textbf{Audiobook} Let customers listen to your audio content.
\item \textbf{Course or tutorial} Sell a single lesson or teach a whole cohort of students.
\item \textbf{Membership} Start a membership business around your fans.
\item \textbf{Physical good} Sell anything that requires shipping something.
\item \textbf{E-book} Offer a book or comic in PDF, ePub, and Mobi formats.
\item \textbf{Podcast} Make episodes available for streaming and direct downloads.
\end{itemize}

It is extremely easy to create an account and put your content up for sale here.

\subsection*{What to write about?}
Your Ebook does not have to be a 100-page book. I have seen people making consistent sales with their books of just around 40 pages, but the topic has to be something they are interested in.

These are the steps I would recommend:

\subsubsection*{1 - Validate your idea for an Ebook}
Ask yourself this:

\begin{itemize}
\item \textit{Which cybersecurity articles on Medium got the most interest and feedback?}
\item \textit{Which of my YouTube videos got the most likes and comments?}
\end{itemize}

Medium and YouTube are a great way to find out which topics are hot and will potentially get you a sale.

You can even run a small poll on LinkedIn or YouTube and ask your network. All of this will help you find that topic before you start writing.

\subsubsection*{2 — Write your eBook}
Writing an eBook is quite easy to do nowadays. These are the most basic tools you need:

\begin{itemize}
\item Microsoft Word or Google docs
\item Grammarly (even the free version will do)
\end{itemize}

\subsubsection*{3 — How to make your eBook look professionally made}
The best and free way to make your eBook look professionally made is to use Canva which is an awesome graphic design website. It has built in templates that you can use to format your eBooks and make them look extremely professionally made. Most of us are not professional designs (nor we can afford them!) so Canva is great at doing the heavy lifting for you.

Below is a ready-made template from Canva where you can just copy your eBook into.

\subsubsection*{4 — How to sell your eBook}
Once you have written and formatted it. You can start hosting it on Gunroad.

Gunroad provides you with a nice hosting page for your eBook but how do you get traffic to it?

This is where your Medium and YouTube channels come in.

You can create a Medium article on the same topic as your eBook and mention that you have a more detailed eBook available for sale on Gunroad.

Similarly, you can create a YouTube video on the same topic and mention your eBook in the video description.

\subsection*{My earnings from Gunroad}
I have sold a few eBooks on Gunroad and the results have been pretty good. This is passive income as I have put in the effort once and the money has been coming in since then.

\begin{itemize}
\item \textbf{Total sales} 11
\item \textbf{Total revenue} \$1,100.00
\item \textbf{Gunroad fees} \$100.00
\item \textbf{Net earnings} \$1,000.00
\end{itemize}

\subsection*{Action items}
These are the action items I need you to take:

\begin{itemize}
\item Create a Gunroad account
\item Find a topic for your eBook from your Medium articles or YouTube videos
\item Write your eBook and format it using Canva
\item Put it up on Gunroad and start promoting it on your Medium and YouTube channels
\end{itemize}

\section*{Self-publish a book on KDP}
\subsection*{DIFFICULTY - MEDIUM}
If you have successfully created an eBook on Gunroad, then the next logical step is to create a paperback book and sell it on Amazon KDP.

\subsection*{What is KDP?}
KDP stands for Kindle Direct Publishing and is a platform by Amazon that allows you to self-publish your books and sell them on Amazon.

\subsection*{Why self-publish on KDP?}
Self-publishing on KDP has a lot of advantages over traditional publishing:

\begin{itemize}
\item \textbf{Complete creative control} You decide the content, cover, and pricing.
\item \textbf{Higher royalties} You earn up to 70\% royalties on sales.
\item \textbf{Global reach} Your book can be available on Amazon worldwide.
\item \textbf{Print on demand} No need to print and store physical copies.
\item \textbf{Fast publishing} Your book can be live on Amazon within 72 hours.
\end{itemize}

\subsection*{How to publish on KDP?}
The process is straightforward:

\subsubsection*{1. Create a KDP account}
Go to \url{https://kdp.amazon.com} and create your account.

\subsubsection*{2. Prepare your manuscript}
Format your book properly. You can use Microsoft Word or free tools like Kindle Create to format your book.

\subsubsection*{3. Design a cover}
You can use Amazon's cover creator tool or design your own cover using Canva or hire a designer from Fiverr.

\subsubsection*{4. Set your price and publish}
Choose your pricing and distribution options, then publish your book.

\subsection*{My experience with KDP}
I've published several books on KDP and the process has been rewarding both financially and personally. The passive income from book sales adds up over time.

\subsection*{Action items}
\begin{itemize}
\item Create a KDP account
\item Choose a topic from your successful eBook or course content
\item Format your manuscript
\item Design a cover
\item Publish your book
\item Promote it through your existing channels
\end{itemize}

\section*{Freelancing on Fiverr}
\subsection*{DIFFICULTY - MEDIUM}
Fiverr is a great platform to offer your cybersecurity services as a freelancer. You can offer services like:

\begin{itemize}
\item Security assessments
\item Vulnerability scanning
\item Security consulting
\item Policy writing
\item Training and awareness
\end{itemize}

\subsection*{How to get started on Fiverr}
\begin{enumerate}
\item Create a seller account
\item Research what services are in demand
\item Create compelling gig offers
\item Set competitive pricing
\item Deliver excellent service to get reviews
\end{enumerate}

\subsection*{My Fiverr experience}
I've found Fiverr to be a great way to pick up consulting work and build my professional network.

\section*{1-1 Bookings}
\subsection*{DIFFICULTY - LOW}
Once you've established your expertise through your content, you can offer 1-1 consulting sessions. Platforms like Calendly make it easy to manage bookings.

\section*{Self-Hosted Courses}
\subsection*{DIFFICULTY - HIGH}
For maximum control and profits, you can create your own website and host courses using platforms like Teachable or Thinkific.

\section*{Time to Earn}
Now that you have multiple income streams set up, remember that consistency is key. Keep creating content, engaging with your audience, and improving your offerings.

The ice cube will melt at 32 degrees - your efforts will compound over time!

\end{document}
